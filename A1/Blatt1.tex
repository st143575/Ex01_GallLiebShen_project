\documentclass[12pt,a4paper]{article}
\usepackage[german]{babel}
\usepackage[utf8]{inputenc}
\begin{document}

\begin{titlepage}
	\begin{center}
	\LARGE \textbf{Datenstrukturen und Algorithmen}\\
	[0.5in]
	\Large \textbf{\"Ubungsblatt 1}\\
	[15cm]
	\end{center}
	\begin{flushright}
	\noindent Robert Gall, 3408913, robert@gall.cc\\
		       Chong Shen, 3111514, shenchong123@yahoo.com\\
		       David Lieb, 3408382, st161483@stud.uni-stuttgart.de\\
	\end{flushright}

\end{titlepage}


\newpage

\noindent \textbf{Aufgabe 1} \\
\textit{(a) sequenzielle Suche}\\
\indent Schritt 1:\quad \fbox{11}\ 22\ 45\ 47\ 66\ 51\ 67\ 72\ 79\ 80\ 81\ 86\ 87\ 88\ 97\\
\indent Schritt 2:\quad 11\ \fbox{22}\ 45\ 47\ 66\ 51\ 67\ 72\ 79\ 80\ 81\ 86\ 87\ 88\ 97\\
\indent Schritt 3:\quad 11\ 22\ \fbox{45}\ 47\ 66\ 51\ 67\ 72\ 79\ 80\ 81\ 86\ 87\ 88\ 97\\
\indent Schritt 4:\quad 11\ 22\ 45\ \fbox{47}\ 66\ 51\ 67\ 72\ 79\ 80\ 81\ 86\ 87\ 88\ 97\\
\indent Schritt 5:\quad 11\ 22\ 45\ 47\ \fbox{66}\ 51\ 67\ 72\ 79\ 80\ 81\ 86\ 87\ 88\ 97\\
\indent Schritt 6:\quad 11\ 22\ 45\ 47\ 66\ \fbox{51}\ 67\ 72\ 79\ 80\ 81\ 86\ 87\ 88\ 97\\
\\
\indent \textit{bin\"are Suche} \\
\indent Schritt 1: \quad [11\ 22\ 45\ 47\ 66\ 51\ 67\ \fbox{72}\ 79\ 80\ 81\ 86\ 87\ 88\ 97]\\
\indent Schritt 2: \quad [11\ 22\ 45\ \fbox{47}\ 66\ 51\ 67]\ 72\ [79\ 80\ 81\ 86\ 87\ 88\ 97]\\
\indent Schritt 3: \quad [11\ 22\ 45]\ 47\ [66\ \fbox{51}\ 67]\ 72\ 79\ 80\ 81\ 86\ 87\ 88\ 97\\
\\
\textit{(b) sequenzielle Suche}\\
\indent Schritt 1:\quad \fbox{12}\ 15\ 21\ 22\ 24\ 31\ 40\ 48\ 55\ 59\ 71\ 88\ 91\ 96\ 97\\
\indent Schritt 2:\quad 12\ \fbox{15}\ 21\ 22\ 24\ 31\ 40\ 48\ 55\ 59\ 71\ 88\ 91\ 96\ 97\\
\indent Schritt 3:\quad 12\ 15\ \fbox{21}\ 22\ 24\ 31\ 40\ 48\ 55\ 59\ 71\ 88\ 91\ 96\ 97\\
\indent Schritt 4:\quad 12\ 15\ 21\ \fbox{22}\ 24\ 31\ 40\ 48\ 55\ 59\ 71\ 88\ 91\ 96\ 97\\
\indent Schritt 5:\quad 12\ 15\ 21\ 22\ \fbox{24}\ 31\ 40\ 48\ 55\ 59\ 71\ 88\ 91\ 96\ 97\\
\indent Schritt 6:\quad 12\ 15\ 21\ 22\ 24\ \fbox{31}\ 40\ 48\ 55\ 59\ 71\ 88\ 91\ 96\ 97\\
\\
\indent \textit{bin\"are Suche} \\
\indent Schritt 1:\quad [12\ 15\ 21\ 22\ 24\ 31\ 40\ \fbox{48}\ 55\ 59\ 71\ 88\ 91\ 96\ 97]\\
\indent Schritt 2:\quad [12\ 15\ 21\ \fbox{22}\ 24\ 31\ 40]\ 48\ [55\ 59\ 71\ 88\ 91\ 96\ 97]\\
\indent Schritt 3:\quad [12\ 15\ 21]\ \fbox{22}\ [24\ 31\ 40]\ 48\ 55\ 59\ 71\ 88\ 91\ 96\ 97\\
\indent Schritt 4:\quad 12\ 15\ 21\ 22\ [24]\ \fbox{31}\ [40]\ 48\ 55\ 59\ 71\ 88\ 91\ 96\ 97\\\\


\noindent \textbf{Aufgabe 4} \\
Fall 1:\ F"ur aufsteigend sortierte Daten:\\
\indent\indent InsertionSort,\ BubbleSort\\
In der Aufgabestellung steht, dass die Daten am schnellsten aufsteigend sortiert werden sollen. Im Fall 1 sind die Daten schon aufsteigend sortiert, also im besten Fall. Deswegen haben InsertionSort und BubbleSort die kleinste Anzahl der Vergleiche, und zwar jeweils circa n Vergleiche.\\\\
Fall 2:\ F"ur absteigend sortierte Daten:\\
\indent\indent MergeSort\\
Im Fall 2 sind die Daten absteigend sortiert und gerade umgekehrt zur Anforderung der Aufgabe. In diesem Fall hat MergeSort die kleinste Anzahl der Vergleiche $n\log_{2}n$\ .\\\\
Fall 3:\ F"ur unsortierte Daten:\\
\indent\indent MergeSort, Quicksort\\
F\"ur unsortierte Daten haben MergeSort und Quicksort loglineare Komplexit\"at, und zwar $n\log_{2}n$ bei MergeSort und $1,386n\log_{2}n$ bei QuickSort.













\end{document}